%%%%%%%%%%%%%%%%%%%%%%%%%%%%%%%%%%%%%%%%%
% The Legrand Orange Book
% LaTeX Template
% Version 2.0 (9/2/15)
%
% This template has been downloaded from:
% http://www.LaTeXTemplates.com
%
% Mathias Legrand (legrand.mathias@gmail.com) with modifications by:
% Vel (vel@latextemplates.com)
%
% License:
% CC BY-NC-SA 3.0 (http://creativecommons.org/licenses/by-nc-sa/3.0/)
%
% Compiling this template:
% This template uses biber for its bibliography and makeindex for its index.
% When you first open the template, compile it from the command line with the 
% commands below to make sure your LaTeX distribution is configured correctly:
%
% 1) pdflatex main
% 2) makeindex main.idx -s StyleInd.ist
% 3) biber main
% 4) pdflatex main x 2
%
% After this, when you wish to update the bibliography/index use the appropriate
% command above and make sure to compile with pdflatex several times 
% afterwards to propagate your changes to the document.
%
% This template also uses a number of packages which may need to be
% updated to the newest versions for the template to compile. It is strongly
% recommended you update your LaTeX distribution if you have any
% compilation errors.
%
% Important note:
% Chapter heading images should have a 2:1 width:height ratio,
% e.g. 920px width and 460px height.
%
%%%%%%%%%%%%%%%%%%%%%%%%%%%%%%%%%%%%%%%%%

%----------------------------------------------------------------------------------------
%	PACKAGES AND OTHER DOCUMENT CONFIGURATIONS
%----------------------------------------------------------------------------------------

\documentclass[11pt,fleqn]{book} % Default font size and left-justified equations


\usepackage{mathrsfs}
\usepackage{amsbsy}

%----------------------------------------------------------------------------------------

\input{structure} % Insert the commands.tex file which contains the majority of the structure behind the template

\begin{document}

%----------------------------------------------------------------------------------------
%	TITLE PAGE
%----------------------------------------------------------------------------------------

\begingroup
\thispagestyle{empty}
\begin{tikzpicture}[remember picture,overlay]
\coordinate [below=12cm] (midpoint) at (current page.north);
\node at (current page.north west)
{\begin{tikzpicture}[remember picture,overlay]
\node[anchor=north west,inner sep=0pt] at (0,0) {\includegraphics[width=\paperwidth]{background}}; % Background image
\draw[anchor=north] (midpoint) node [fill=ocre!30!white,fill opacity=0.6,text opacity=1,inner sep=1cm]{\Huge\centering\bfseries\sffamily\parbox[c][][t]{\paperwidth}{\centering Probability Theroy based on Measure Theory\\[15pt] % Book title
{\Large STAT 517}\\[20pt] % Subtitle
{\huge Dr. John Smith}}}; % Author name
\end{tikzpicture}};
\end{tikzpicture}
\vfill
\endgroup

%----------------------------------------------------------------------------------------
%	COPYRIGHT PAGE
%----------------------------------------------------------------------------------------

\newpage
~\vfill
\thispagestyle{empty}

\noindent Copyright \copyright\ 2013 John Smith\\ % Copyright notice

\noindent \textsc{Published by Publisher}\\ % Publisher

\noindent \textsc{book-website.com}\\ % URL

\noindent Licensed under the Creative Commons Attribution-NonCommercial 3.0 Unported License (the ``License''). You may not use this file except in compliance with the License. You may obtain a copy of the License at \url{http://creativecommons.org/licenses/by-nc/3.0}. Unless required by applicable law or agreed to in writing, software distributed under the License is distributed on an \textsc{``as is'' basis, without warranties or conditions of any kind}, either express or implied. See the License for the specific language governing permissions and limitations under the License.\\ % License information

\noindent \textit{First printing, March 2013} % Printing/edition date

%----------------------------------------------------------------------------------------
%	TABLE OF CONTENTS
%----------------------------------------------------------------------------------------

\chapterimage{chapter_head_1.pdf} % Table of contents heading image

\pagestyle{empty} % No headers

\tableofcontents % Print the table of contents itself

\cleardoublepage % Forces the first chapter to start on an odd page so it's on the right

\pagestyle{fancy} % Print headers again

%----------------------------------------------------------------------------------------
%	PART
%----------------------------------------------------------------------------------------

\part{Part One}

%----------------------------------------------------------------------------------------
%	CHAPTER 1
%----------------------------------------------------------------------------------------

\chapterimage{chapter_head_2.pdf} % Chapter heading image

\chapter{Probability Measure}

\section{Overview}

\section{Probability on a Field}\index{Probability on a Field}
% Section Two in Billingsly

\begin{definition}[$\Omega$] Non emtpy set. 
	
\end{definition}

\begin{definition}[Paving] A collection of a subset of $\Omega$ is a paving. 
	
\end{definition}

\begin{definition}[Field] A \textbf{field} $\mathscr{F}$ is a paving satisfying
	\begin{enumerate}[label = (\roman*)]
		\item $\Omega \in \mathscr{F}$
		\item $A \in \mathscr{F} \Rightarrow A^C \in \mathscr{F}$
		\item $A, B, \in \mathscr{F}, \Rightarrow A \cup B \in \mathscr{F}$
	\end{enumerate}
	
\end{definition}

\textbf{Derived Properties about a Field }

\begin{itemize}
	\item $\emptyset \in \mathscr{F}$ 
		(by (i) and (ii): 
		\begin{align*}
			\Omega \in \mathscr{F} &\Rightarrow \Omega^C \in \mathscr{F}\\
			&\Rightarrow \emptyset \in \mathscr{F})
		\end{align*}
			
	\item (i) can be replaced by "$\mathscr{R}$ is nonempty" because,\\ 
		Let $A \in \mathscr{F}, $
			\begin{align*}
				&\Rightarrow A^c \in \mathscr{F}\\
			&\Rightarrow A^C \cup A \in \mathscr{F}\\
			&\Rightarrow \Omega \in \mathscr{F}\\
			\end{align*}
			
	\item $A \in \mathscr{F}, B \in \mathscr{F}, \Rightarrow, A\cap B \in \mathscr{F}$ 
	because, \\
		\begin{align*}
		&(A\cap B)^C = A^C \cup B^C (DeMorgan's Law)\\
		&A \cap B = (A^C \cup B^C)^C	
		\end{align*}
		

	\item $A_1, \dots, A_m \in \mathscr{F} \Rightarrow A_1 \cup, \dots, \cup A_m \in \mathscr{F} $ (mathematical induction)
	\item$A_1, \dots, A_m \in \mathscr{F} \Rightarrow A_1 \cap, \dots, \cap A_m \in \mathscr{F} $ 
\end{itemize}



\begin{definition}[$\sigma$-Field] Similar to the definition of a field except for (iii). A paving satisfying 
	\begin{enumerate}[label = (\roman*)]
		\item $\Omega \in \mathscr{F}$
		\item $A \in \mathscr{F} \Rightarrow A^C \in \mathscr{F}$
		\item $A_1 \in \mathscr{F}, \dots, A_m \in \mathscr{F} $\\ $\displaystyle\cup_{k=1}^m A_k \in \mathscr{F}$ (finite additvity)
	\end{enumerate}
\vspace{5mm}

If we replace (iii) from before by (iii') here:

For $A_1 \in \mathscr{F}, \dots, A_m \in \mathscr{F}$ 

$$\cup_{k=1}^\infty A_k \in \mathscr{F}$$

then $\mathscr{F}$ is called a $\sigma$\textbf{-field}. 
\end{definition}


	


\textbf{Derived Facts}

\begin{itemize}
	\item Again, (i) can be repalced by $\mathscr{F}$ no empty, 
				(iii) can be replaced  $A_1 \in \mathscr{F}, \dots, A_m \in \mathscr{F} $
\end{itemize}

\begin{example}
	$\Omega = (0, 1]$ (from now on all intervals are left open, right closed)\\

	\begin{remark}
		Recall that $\sigma$-fields are generated by fields. Fancy scripts denote a $\sigma$-field. Fancy scripts with a zero subscript denote a field.
	\end{remark}

	$\mathscr{B}_0$ is the collection of all finite union of disjoint intervals. Asside: something that we can consider easily, but computers cannot. (e.g. think of all stars, etc)

	\begin{figure}[h]
	\centering\includegraphics[scale=0.15]{notes1.jpg}
	\caption{Finite unioin of three disjoint intervals.}
	\end{figure}

	Then $\mathscr{B}_0$ is a field. 

	\begin{enumerate}[label = (\roman*)]
		\item (0, 1] $\in \mathscr{B}_0$
		\item $A \in \mathscr{B}_0 \Rightarrow A^c \in \mathscr{B}_0$ 
		\begin{figure}[h]
		\centering\includegraphics[scale=0.15]{notes2.jpg}
		\caption{A and complement of A.}
		\end{figure}

		\item $A \in \mathscr{B}_o, B \in \mathscr{B}_o \Rightarrow A \cup B \in \mathscr{B}_o$ 

		\begin{figure}[h]
		\centering\includegraphics[scale=0.15]{notes3.jpg}
		\caption{Union of A and B is still in $\mathscr{B}_o$}
		\end{figure}

	\end{enumerate}


\end{example}


\textbf{Wednesday August 24}

$\mathscr{B}_0 = $ collection of finite unions of disjoin subintervals of (0, 1]. Is a field. \\
\\

\begin{definition}[Power Set]
	A $\sigma$-field is generated by a paving of power set. Let $\Omega$ be a set. The collection of all subsets of $\Omega$ is the power set written as $2^\Omega$.
\end{definition}

\begin{remark}
	Where does this notation come from?

Consider ths case where $\Omega$ is finite

$$\Omega = \{\omega_1, \dots, \omega_n \} $$

Total number of subsets of $\Omega$. 

$\emptyset, 1$ element sets, 2-element sets, $\dots$, n-element ests.

$$() + () + \dots + = (1+1)^n$$

$\#(\mathscr{F})  = 2^{\# \Omega}$, so it seems reasonable to denote $\mathscr{F} = 2^\Omega$. 

It is also easy to show that $2^\Omega$ is a $\sigma$-field. (The largest, even. The smallest: $\{\emptyset, \Omega\}$ which is also a $\sigma$-field.)

$$\{\emptyset, \Omega\} \subseteq \sigma\text{-field} \subseteq 2^\Omega$$ 
\end{remark}

It turns out we can extend notion of lenght from $\mathscr{B}_0$ to $\sigma$-field generated by $\mathscr{B}_o$. \\
\\
Now, let $\mathscr{A}$ be a nonempty paving of $\Omega$. We define 
$$\sigma(\mathscr{A}) = \cap \{\mathscr{B} \subseteq 2^\Omega: \mathscr{B}\text{ is a }\sigma\text{-field}, \mathscr{A} \subseteq \mathscr{B}\} $$

OR rather, the \textit{intersection} of all $\sigma$-fields that contains $\mathscr{A}$. \\
\\
Let 
$$\mathbb{F}(\mathscr{A}) = \{\mathscr{B} \subseteq 2^\Omega: \mathscr{B} \text{ is a } \sigma\text{-field, } \mathscr{B} \supseteq \mathscr{A} \}$$

Then, 
$$\sigma(\mathscr{A}) = \cap \mathscr{B}$$
$$\mathscr{B} \in \mathbb{F}(\mathscr{A}) $$

\textbf{Derived Facts}

$\mathbb{F}(\mathscr{A})$ is nonempty. For example, $2^\Omega$ is a $\sigma$-field and $2^\Omega \supseteq \mathscr{A}$. 

$\cap B$ is a $\sigma$-field. ($B \in \mathbb{F}(\mathscr{A})$)

\begin{remark}
	Get notes about notation/levels.
\end{remark}

\begin{proof} We will prove that indeed $\sigma(\mathscr{A})$ is a $\sigma$ -field. Recall that we have three conditions above for $\sigma$-field.\\
\\
	\begin{enumerate}[label = (\roman*)]
		\item $$\Omega \in \sigma(\mathscr{A})$$ 
			$$\Omega \in \cap_{B \in \mathbb{F}(\mathscr{A})} B $$
			Because: B is $\sigma$-field, $A \in B$, $\forall B \in \mathbb{F}(\mathscr{A})$.
		\item 
			% \begin{align*}
			% 	A \in \cap B &\Rightarrow A \in B \forall B \in\mathbb{F}(\mathscr{A})\\
			% 	&\Rightarrow A^C \in B , \forall B  \in \mathbb{F}(\mathscr{A})\\ 
			% 	&\Rightarrow A^C \in \cap_{B \in \mathbb{F}(\mathscr{A}) B\\
			% \end{align*}

		\item $A_1, \dots, \in \cap_{B \in \mathbb{F}(\mathscr{A})} B, \forall  B  \in \mathbb{F}(\mathscr{A})$

		$\Rightarrow \cup^\infty_{n =1} A_n \in B, \forall B \in \mathbb{F}(\mathscr{A})$

	\end{enumerate}

	So, $\sigma(\mathscr{A}$ is a $\sigma$-field, we call it the $\sigma$-field, generated by $\mathscr{B}_o$. We know how tot assign lenth to members of $\mathscr{B}_o$, we now show the assignment can be extended to $\sigma(\mathscr{B}_o)$ 


\end{proof}

\begin{example}
	 Let $\mathscr{I}$ be the collection of \textit{all} subintervals of (0,1].\\
	\\
	 Note that $\mathscr{I}$ is a smaller collection than $\mathscr{B}_0$ since $\mathscr{B}_0$ can have numerous different combinations of the sets. 

	 Let

	$$\mathscr{B} = \sigma(\mathscr{I})$$. 

	This is a Borel-$\sigma$-field. (a member of $\mathscr{B}$ in Borel set.)

	It turns out

	$$\sigma(\mathscr{I}) = \sigma(\mathscr{B}_o)  $$

This is because $\sigma(\mathscr{I})$ is a $\sigma$-field. 

So, 
	\begin{align*}
		\sigma(\mathscr{I}) &\supseteq \mathscr{B}_o\\
		\sigma(\mathscr{I}) &\supseteq \sigma(\mathscr{B}_o)
	\end{align*}

Also, 
\begin{align*}
		\mathscr{I} &\subseteq \mathscr{B}_o\\
		\sigma(\mathscr{I}) &\subseteq \sigma(\mathscr{B}_o)
	\end{align*} 

Thus, 
\begin{align*}
	\sigma(\mathscr{I}) &= \sigma(\mathscr{B}_o)
\end{align*}
\end{example}




\begin{definition}[Probability Measure]

Probablity measures on field. Suppose $\mathscr{F}$ is a field on a nonempy set $\Omega$. A probability measure is a function $P:\mathscr{F} \rightarrow \mathbb{R}$. 

\begin{enumerate}[label = (\roman*)]
	\item $0 \leq P(A) \leq 1, \forall A \in \mathscr{F}$
	\item $P(\emptyset) = 0$, $P(\Omega) = 1$
	\item If $A_1, \dots$ are disjoint emembers of $\mathscr{F}$ and $\cup A_n \in \mathscr{F}$ then we have countable additivity:

	$$P (\cup A_n) = \displaystyle\sum^\infty_{n=1} P(A_N) $$
\end{enumerate}

\begin{remark}
	Note that (iii) also implies finite additivity. Prove by adding infinite empty sets on end. 
\end{remark}

	
\end{definition}


If $\Omega$ is nonempty set. 
And $\mathscr{F}$ is a $\sigma$-field on $\Omega$.
And P is a probability measure on $\mathscr{F}$.

Then ($\Omega, \mathscr{F}, P$) is called a \textbf{probability space.}

And ($\Omega, \mathscr{F}$) is called a \textbf{measurable space.}


\begin{remark}
	If $A \subseteq B$, then $P(A) \leq P(B)$. This is because we may write B as

	$$B = A \cup (B\setminus A) $$
\end{remark}

\begin{remark}
	$$P(A) + P(B) = P(A\cup B) + P(A \cap B)$$


\end{remark}

\textbf{Friday August 26}
\\

Recall, 

Probability measure on a field, $\mathscr{F}_0$.

\begin{itemize}
	\item $P(A) + P(B) = P(A\cup B) + P(A \cap B)$

	% GET VINN DIAGRAM
	\begin{itemize}
		\item $P(A) = P(AB^C) + P(A B)$
		\item $P(B) = P(B A^C) + P(AB)$
		\item $P(A) + P(B) = P(AB^C) + P(BA^C) + 2P(AB)$
		\item $P(A \cup B) = P(AB^C) + P(BA^C) + P(AB)$ 
	\end{itemize}
	
	\item $P(A \cup B) = P(A) + P(B) - P(AB)$
		By induction, we can prove if $A_1, \dots A_n$,

		$$P(\displaystyle \cup^n_{k=1} A_k) = \displaystyle \sum^n_{k=1} P(A_k) - \displaystyle \sum_{i<j} P(A_iA_j) +
		\displaystyle \sum_{i<j<k} A_iA_j) + \dots + (-1)^{n+1} P(A_1, \dots, A_n)  $$

		Inclusion- Exclusion Formula

	\item If $A_1, \dots A_n \in \mathscr{F}$,

		$$B_1 = A_1 $$
		$$B_2 = A_2 \setminus A_1$$
		$$\vdots $$

		Then, 
		$$\displaystyle \cup^n_{k=1} A_k = \displaystyle \cup^n_{k=1} B_k $$

		but the $B_i$ are disjoint. Also $A_K \subseteq B_k \forall k=1, \dots, n$.

		$$ P(\displaystyle \cup^n_{k=1} A_k) = P(\displaystyle \cup^n_{k=1} B_k) = \displaystyle \sum^n_{k=1} B_k \leq \displaystyle \sum^n_{k=1} A_k$$

		Thus, $P(\displaystyle \cup^n_{k=1} A_k) \leq \displaystyle \sum^n_{k=1} A_k$. Finite subadditivity. 

\end{itemize}

Some conventions, 

If $A_1, \dots$ is a sequence of sets, we say $A_n \uparrow A$ if 

\begin{enumerate}
 	\item $A_1 \subseteq A_2 \subseteq \dots$
 	\item $\displaystyle \cup^\infty_{k=1} A_k = A$
 \end{enumerate} 
\vspace{5mm}
 If $A_1, \dots$ is a sequence of sets, we say $A_n \downarrow A$ if 

\begin{enumerate}
 	\item $A_1 \supseteq A_2 \supseteq \dots$
 	\item $\displaystyle \cap^\infty_{k=1} A_k = A$
 \end{enumerate} 

 \begin{theorem}
 	If $P$ is a probability measure on a field $\mathscr{F}$ Then, 

 	\begin{enumerate}
 		\item Continuity from below.

 		If $A_n \in \mathscr{F} \quad \forall n, A \in \mathscr{F}$
 		$$ A_n \uparrow A$$

 		then $$P(A_n) \uparrow P(A)$$

 		\item Continuity from above.

 		If $A_n \in \mathscr{F} \quad \forall n. A \in \mathscr{F}$  
 			$$A_n \downarrow A$$
 		 then $$P(A_n) \downarrow P(A)$$

 		\item Countable subadditivity.

 		If $A_n \in \mathscr{F} \quad \forall n. \displaystyle \cup^\infty_{k=1} A_k \in \mathscr{F}$ then 

 		$$P(\displaystyle \cup^\infty_{n=1} A_k) \leq \displaystyle \sum^\infty_{n=1} P(A_k)$$
 	\end{enumerate}
 \end{theorem}

 \begin{proof}

 \begin{enumerate}
 	\item If $A_1, \dots A_n \in \mathscr{F}$,

		$$B_1 = A_1 $$
		$$B_2 = A_2 \setminus A_1$$
		$$B_3 = A_3 \setminus A_2$$
		$$\vdots $$

		then, $B_1, \dots$ are disjoint. 

		$$\displaystyle \cup^\infty_{n=1} A_n = \displaystyle \cup^\infty_{n=1} B_n $$

		$\begin{aligned}
			P(A) &= P(\displaystyle \cup^\infty_{n=1} A_n) \\
		&= P(\displaystyle \cup^\infty_{n=1} B_n ) \\
		&= \displaystyle \sum^\infty_{n=1} P(B_n) \\
		&= \lim_{n \rightarrow \infty} \displaystyle \sum^\infty_{n=1} P(B_n)\\ 
		&= \lim_{n \rightarrow \infty} P(A_n)
		\end{aligned}
		$

	\item $A_n \downarrow A \Leftrightarrow A_n^C \uparrow A^C$

	But by (1), 

	$$P(A_n^C) \uparrow P(A^C)$$
	$$1 - P(A_n) \uparrow 1 - P(A)$$
	$$P(A_n) \downarrow P(A)$$


	\item By finite subadditivity, 

	$$ P(\displaystyle \cup^n{k=1} A_k) \leq \displaystyle \sum^n{k=1} P(A_k) \leq \displaystyle \sum^\infty_{n=1} P(A_n)$$

	But since, by (1), because

	$$\displaystyle \cup^n_{k=1} A_k \uparrow \displaystyle \cup^\infty_{n=1} A_n$$

	$$P(\displaystyle \cup^n_{k=1} A_k) \uparrow P(\displaystyle \cup^\infty_{n=1} A_n)$$

	So, 

	$$P(\displaystyle \cup^\infty_{n=1} A_n) \leq
		 \displaystyle \sum^\infty_{n=1} P(A_n)$$


 \end{enumerate}
\end{proof}

\begin{remark}
	$A \in \mathscr{F} =$ "A is F-set".
\end{remark}


\section{Extention of Probability Measure to a $\sigma$-field}\index{Exten. Prob Measure to $\sigma$-field}

Let $f$ be a function $f: D\rightarrow R$. 

Let $\tilde{D}$ be another set such that 

$$D \subseteq \tilde{D} $$

An extantion of $f$ onto  $\tilde{D}$ is 

$$\tilde{f}: \tilde{D} \rightarrow R $$

Such that $f(x) = \tilde{f}(x) \forall x \in D$

$\tilde{f}$ is an extention of $f$ on D. 

We say $f$ has unique extention, $\tilde{f}$ onto $\tilde{D}$ if 

\begin{enumerate}
	\item $\tilde{f}$ is an extension of $f$ to $\tilde{D}$.

	\item if $g$ is another extension of $f$ to $\tilde{D}$ then $\tilde{f} = g$ on $D$.
\end{enumerate}


\begin{theorem}
	A probability measure on a field has a unique extension on the $\sigma$-field generated by this field. 

		This means that if $\mathscr{F}_0$ is a field, and $P$ is a probability measure on $\mathscr{F}_0$, then there exists a probability measure, $Q$ on $\sigma(\mathscr{F})$ such that 
		$$Q(A) = P(A)\quad \forall A \in \mathscr{F}_0$$

		Moreover, if $\tilde{Q}$ is another probability measure on $\sigma(\mathscr{F}_0)$ such that $\tilde{Q} = P(A) \quad \forall A \in \mathscr{F}$ then $$\tilde{Q} = Q$$. 
\end{theorem}

	\begin{remark}
		The proof of this theorem will come after several definitions and lemmas. 
	\end{remark}

	\textbf{Outer Measure} $P^*: 2^\Omega \rightarrow \mathbb{R}$  

	For any $A \in 2^\Omega$ ($A \subseteq \Omega$)

	$$P^*(A) = \inf \{\displaystyle \sum_{n=1}^\infty P(A_n): A_1, A_2, \dots \text{is a sequence of } \mathscr{F}_0 \text{ sets, } A \subseteq \cup^\infty_{n=1} A_n\} $$

	$P^*$ is a measure out until $\mathscr{M}$, but it is only a function beyond that on $2^\Omega$.\\

\textbf{Inner Measure}

$P_*(A) = 1 - P^*(A)$ 

\vspace{5mm}

Define the paving $\mathscr{M}$ as followes
$$\mathscr{M} = \{ A \in 2^\Omega:
		E \in 2^\Omega,
		P^*(E) = P^*(E\cap A) + P^*(E \cap A^C) \}$$

	Idea: we came up with this $\mathscr{M}$ such that $P^*$ behaves as a measure. It will turn out to be that $\mathscr{M}$ is a $\sigma$-field that contains $\sigma(\mathscr{F}_0)$.\\

\textbf{Monday August 29}\\

$P^*$ satisfies the following probabilities:

\begin{enumerate}[label = (\roman*)]
	\item $P^*(\emptyset) = 0$
	\item $P^*(A) \geq 0 \quad \forall A \in 2^\Omega$
	\item $A \subseteq B \Rightarrow P^*(A) \subseteq P^*(B)$
	\item $P^*(\cup^\infty_{n=1} A_n) \leq \displaystyle \sum^\infty_{n=1} P^*(A_n))$
\end{enumerate}

\begin{proof}
	
	\begin{enumerate}[label = (\roman*)]
		\item Take $\{\emptyset, \emptyset, \dots \}$. 

		$$\emptyset \in \mathscr{F}_0, \quad \emptyset \cup^\infty_{n=1} \emptyset $$

		So, \\
		$$P^*(\emptyset) \leq \displaystyle \sum^\infty_{n=1} P(\emptyset) = 0$$

		Note,

		$$P(A) \geq 0 \quad \forall A$$

		So, 

		$$P^*(\emptyset) \geq \emptyset$$ 

		Thus,

		 $$P^*(\emptyset) = \emptyset$$
		\item  Already done as part of (i).

		\item  Let $A \subseteq B$

		$$P^*(A) = \inf\{\displaystyle \sum^\infty_{n=1} P(A_n), A_n \in \mathscr{F}_0, A \subseteq \cup A_n \} $$

		Now, if $B_1, \dots \in \mathscr{F}_0 \subseteq \cup B_n$

		Then, 
		$$A \subseteq B \subseteq \cup_n B_n $$

		If  $\{ \{B_n\}^\infty_{n=1}: B_n \in \mathscr{F}_0, B \subseteq \cup_n B_n \} \subseteq \{ \{A_n\}^\infty_{n=1}: A_n \in \mathscr{F}_0, A \subseteq \cup_n A_n \}$

		Or in short, Collection 1 $\subseteq$ Collection 2.\\

		So the inf of a larger set is smaller than (or equal to) the inf of a smaller set.\\
		
		So, 

		$P^*(A) = \inf\{\displaystyle \sum^\infty_{n=1} P(A_n), A_n \in  \text{ collection \#1}\} \leq P^*(B) = \inf\{\displaystyle \sum^\infty_{n=1} P(B_n), A_n \in  \text{ collection \#2}\} = P^*(B)$

		\item Want 

		$$P^*(\cup_n A_n) \leq \displaystyle\sum_n P^*(A_n) $$

		$P^*(A_n) = \inf \{\displaystyle \sum_{n=1}^\infty P(A_n): A_{nk} \in \mathscr{F}_0,  A \subseteq \cup_{k} A_{nk}\}$\\

		Let $\epsilon > 0$, by defnition of  there exists, 

		$$ \{B_n\}^\infty_{n=1}  $$ such that

		$$\displaystyle \sum^\infty_{k=1} P(B_{nk}) \leq P^*(A_n) + \frac{\epsilon}{2^n} $$

		So, 

		$$\cup_n A_n \subseteq \cup_{n,k} B_{nk} $$

		and,\\

		$\begin{aligned}
			P^*(\cup_n A_n) &\leq \displaystyle \sum_{n,k} P(B_{nk})\\ 
			&< \displaystyle \sum_n P^* (A_n) + \sum_n (\epsilon 2^{-n})\\
			P^*(\cup A_n) &< \sum_n P^* (A_n) + \epsilon \quad \forall \epsilon > 0
		\end{aligned}$

		Simply put, \\
		$ b$

		So, 

		$$P^*(\cup_n A_n) \leq \sum_n P^*(A_n) $$
 	\end{enumerate}
\end{proof}

By definition, $A \in \mathscr{M}$ if and only if $P^*(EA) + P^*(EA^C) = P^*(E)$. \\

We know that $P^*$ is subadditive. \\

So, by subadditivity we know, 

$$P^*(E) \leq P^*(AE) + P^*(A^C E) $$

Therefore, to show $A \in \mathscr{M}$ we only need to show 

$$P^*(E) \geq P^*(AE) + P^*(A^C E) $$


$\mathscr{M}$ is defined by $P^*$ and $P^*$ is defined using $\mathscr{F}_0$ so $\mathscr{M}$ is indireclty tied to $\mathscr{F}_0$.\\

\textbf{Lemma 1.} $\mathscr{M}$ is a field.

\begin{proof}


\begin{enumerate}[label = (\roman*)]
	\item $\Omega \in \mathscr{M}$\\

	$$\begin{aligned}
			A &= \Omega\\
			P^*(\emptyset) &= 0\\
			P^*(E) + P^*(\emptyset) &= P^*(E)
		\end{aligned}$$

	\item $A \in \mathscr{M} = A^C \in \mathscr{M}$\\

	$$\begin{aligned}
		P^*(E) &= P^*(EA) + P^*(A^C E)\\
		&= P^*(EA^C) + P^*(A E)\\
		&= P^*(EA^C) + P^*((A^C)^C E)
	\end{aligned}$$

	\item $A, B \in \mathscr{M} \rightarrow A \cap B \in \mathscr{M}$\\

	$B \in \mathscr{M} \Rightarrow P^*(E) = P^*(Eb) + P^*(B^C E) \quad \forall E$

	$A \in \mathscr{M} \Rightarrow P^*(BE) = P^*((BE)A) + P^*(A^C (BE))$

	$A \in \mathscr{M} \Rightarrow P^*(B^CE) = P^*((B^CE)A) + P^*(A^C (B^CE))$\\

	Hence, \\
	
	$$P^*(BE) + P^*(B^CE) = P^*((BE)A) + P^*(A^C (BE)) + P^*((B^CE)A) + P^*(A^C (B^CE))$$

	$$\begin{aligned}
		P^*(A^C (BE)) + P^*((B^CE)A) + P^*(A^C (B^CE)) &\geq P^*((A^C BE) \cup (AB^CE)\cup(A^CBE))\\
			&= P^*(E\cap[A^CB\cup  AB^C\cup A^CB^C])\\
			&= P^*(E \cap (AB)^C)
	\end{aligned} $$

	$$\begin{aligned}
		P^*(E) &= P^*(BE) + P^*(B^CE)\\
		&= P^*((BE)A) + (P^*(A^C (BE)) + P^*((B^CE)A) + P^*(A^C (B^CE)))\\
		&\geq P^*(ABE) + P^*(E(AB)^C)
	\end{aligned} $$
	
	 

	So, $A,B \in \mathscr{M}$
\end{enumerate}
\end{proof}

\textbf{Lemma 2.} If $A_1, A_2, \dots$ is a sequence of disjoint $\mathscr{M}$-sets then for each $E \subseteq \Omega$, 
$$P^*(E\cap(\cup_k A_k)) = \displaystyle \sum_k P^*(E \cap A_k) $$

\begin{proof}
	First, prove this statement for finite sequence. 

	$$A_1, \dots, A_n $$

	by mathematical induction. \\
	\\

	If $n=1$ this is 'trivial', 

	$$P^*(E\cap A_1) = P^*(E \cap A_1) $$

	If $n = 2$ we need to show, 

	$$ P^*(E  (A_1 \cup A_n)) = P^*(E A_1) + P^*(E A_2)$$

	Because $A_1 \in \mathscr{M}$, 

	$$P^*(E(A_1 \cup A_2)) = P^*(E(A_1 \cup A_2)) A_1 + P^*(E(A_1 \cup A_2)A_1^2)  $$

	$$E(A_1 \cup A_2) = E(A_1 A_2 \cup A_1 A_2 = EA_1$$

	$$E(A_1 \cup A_2) A_1^C = E(A_1 A_1^C \cup A_2 A_2^C)$$

	So, 

	$$P^*(E(A_1 \cup A_2)) = P^*(EA_1) + P^*(EA_2)$$

Suppose true for n = k. (induction hypothesis) \\
\\
Now we must show for n = k + 1.

$$P^* (E \cap (\cup_{n=1}^{k+1} A_n)) = P^*([E \cap (\cup_{n=1}^{k} A_n)] \cup A_{k+1}) $$

$ (\cup_{n=1}^{k} A_n), A_{k+1}$ are two disjoint sets. Using the n=2 case, 

$$ = \displaystyle \sum_{n=1}^k P^*(E \cap A_n) + P(E \cap A_{k+1}) = \displaystyle \sum_{n=1}^{k+1} P^*(E \cap A_n)  $$

So this is now shown to be true for $\{A_1, \dots, A_n \}$. Next, showtrue for $A_1, \dots in \mathscr{M}$ (disjoint).

Want: 

$$P^*(E \cap (\cup_{n=1}^\infty A_n)) = \displaystyle \sum_{n=1}^{\infty} P^*(E \cap A_n)  $$

Using countable subadditivity, 

$$ P^*(E \cap (\cup_{n=1}^\infty A_n)) = P^*(\displaystyle \cup_{n=1}^{\infty} E \cap A_n) \leq \displaystyle \sum_{n=1}^{\infty} P^*(E \cap A_n)$$

In the meantime, by the monotonicity of $P^*$

$$P^*(E \cap (\cup_{n=1}^\infty A_n)) \geq P^*(E \cap (\cup_{n=1}^m A_n)) =  \displaystyle \sum_{n=1}^{\infty} P^*(E \cap A_n)$$

So, 

$$P^*(E \cap (\cup_{n=1}^\infty A_n)) \geq \lim  \displaystyle \sum_{n=1}^{m} P^*(E \cap A_n)$$


(*), (**) gives us, 

$$P^*(E \cap (\cup_{n=1}^\infty A_n)) = \displaystyle \sum_{n=1}^{\infty} P^*(E \cap A_n)  $$

\end{proof}

\textbf{Wednesday August 31}

(finished proof)\\
\\

\textbf{Lemma 3.}
	
	\begin{enumerate}
		\item $\mathscr{M}$ is a $\sigma$-field
		\item $P^*$ restricted on $\mathscr{M}$ is countably additive. 
	\end{enumerate}

\begin{proof}
	First we show if\\

	\begin{enumerate}
		\item $\mathscr{M}$ is a fieldd
		\item $\mathscr{M}$ is closed under countable disjoint union.
	\end{enumerate}

then $\mathscr{M}$ is a $\sigma$-field.

Let's create disjoints sets, 

$A_n \in \mathscr{M}, n = 1, 2, \dots$
$B_1 = A_1$
$B_2 = A_2 A^C_1$
$\vdots$
$B_n = A_n A_1^C \dots A_{n-1}^C$

$B_1, \dots, B_n \in \mathscr{M}$ (disjoint)


$$\cup^\infty_{n=1} B_n = \cup^\infty_{n=1} A_n $$

But we know that $\cup^\infty_{n=1} B_n \in \mathscr{M}$ so $\cup^\infty_{n=1} A_n \in \mathscr{M}$ and thus $\mathscr{M}$ is a $\sigma$-field.

So it suffices to show that $\mathscr{M}$ is closed under disjoint countable unions.\\
\\
Let $A_1, A_2, \dots$ are disjoins $\mathscr{M}$-sets.\\ 
\\
Let $A = \cup^\infty_{n=1} A_n$.\\ 
\\
Let $F_n = \cup^n{k=1} A_k$.\\ 
\\
Then $F_n \in \mathscr{M}$.\\
\\
So, $\forall E \in 2^\Omega$, 

$$P^*(E) = P^*(E F_n) + P^*(E F_n^C) $$

\begin{align*}
	P^*(E F_n) &= P^*(E(\cup_{k=1}^n A_k))\\
		&= \displaystyle \sum^n_{k=1} P^*(E A_k)\\
P^*(E F_n^C) &\geq P^*(E A^C)  (F_n \subseteq A, F_n^C \supseteq A^C)\\
	\Rightarrow P^*(E) \geq \lim_{n \rightarrow \infty} P^*(E A_k) + P^*(E A^C)\\
		&= \displaystyle \sum^n_{k=1} P^*(E A_k) + P^*(E A^C)\\
		&= P^*(EA) + P^*(E A^C)
\end{align*}
 

\end{proof}

So $A \in \mathscr{M}$ and $\mathscr{M}$ is a $\sigma$-field. \\
\\

Now, let's show $P^*$ is countably additive.

Let $A_1, A_2, \dots$ be disjoint members of $\mathscr{M}$. Then $\forall E \in 2^\Omega$, 

$$P^*(E(\cup^\infty_{n=1} A_n)) = \displaystyle \sum^\infty_{n=1} P(E A_n) $$

Take $E = \Omega$. 

$$P^*(\cup^\infty_{n=1} A_n) = \displaystyle \sum^\infty_{n=1} P( A_n) $$


\textbf{Lemma 4.} $\mathscr{F}_0 \subseteq \mathscr{M}$

\begin{proof}
	Let $A \in \mathscr{F}$.\\
	\\

	Want:

	$$A \in \mathscr{M} $$
$$P^*(E) = P^*(EA) + P^*(E A^C) $$

By definition, there exists $E_n \in \mathscr{F}_0$ 
such that 

$$\displaystyle \sum^\infty_{n=1} P^*(E_n) \leq P^*(E) + \epsilon $$

% \begin{align*}
% 	P^*(EA) &\leq P^*((\cup^\infty_{n=1} E_n)A) \text{ (monotonocity)}\\
% 		&= P^* (\cup^infty_{n=1} (E_n A))\\
% 		&\leq \displaystyle \sum^infty_{n=1}P^* ( (E_n A)) \text{ (countibly subadd)}\\
% 	P^*(E A^C) &\leq \displaystyle \sum^\infty_{n=1} P^*(E_n A^C)\\
% 	P^*(EA) + P^*(E A^C) &\leq 	\displaystyle \sum^\infty_{n=1} P^*(E_n A) + P^*(E_n A^C)\\
% 		&= \displaystyle \sum^\infty_{n=1} P^*(E_n)\\
% 	\text{Recall, } A, E_n \in \mathscr{F}_0\\
% 		&\leq P^*(E) + \epsilon\\
% 	P^*(EA) + P^*(E A^C) & \leq P^*(E) + \epsilon \quad \forall \epsilon\\
% 	\Rightarrow P^*(EA) + P^*(EA^C) &= P^*(E)\\
% 	\Rightarrow A \in \mathscr{M}\\
% 	\mathscr{F}_0 \in \mathsdcr{M}
% \end{align*}
 
\end{proof}

\textbf{Lemma 5.}$$P^*(A) = P(A) \quad \forall A \in \mathscr{F}_0$$

\begin{proof}
	Let $A \in \mathscr{F}_0$. 

	Because,  $A, \emptyset, \emptyset, \dots, \in \mathscr{F}_0$. 

	$$A \subseteq A \cup \emptyset \cup \emptyset \dots $$

	$$P^*(A) \leq P(A) + P(\emptyset) + \dots $$

But, if 

$$A_n \in \mathscr{F}_0$$

$$A \subseteq \cup^\infty_{n=1} A_n $$

$$P^*(A) \leq \displaystyle \sum^\infty_{n=1} P(A_n)$$

$$\Rightarrow P^*(A) \leq \inf \displaystyle \sum^\infty_{n=1} P(A_n) $$

$$ = P^*(A)$$
\end{proof}

\textbf{Friday September 2}

Recall, Extension Theorem. That is, If $\mathscr{F}$ is a field and $P$ is a probability measure, then there exists a measure, $Q$ such that 

	$$Q(A) =  P(A) \quad \forall A \in \mathscr{F}_0$$

\begin{proof}
	By Lemma 5, 

	$P^*(\Omega) = P(\Omega) = 1$
	$P^*(\emptyset) = P(\emptyset) = 0$

	Outline of what we need to show:

	\begin{itemize}
		\item $ 0 \leq M(A) \leq 1$
		\item $M(\emptyset) = 0, \quad M(\Omega)=1$
		\item $M(\cup_n A_n) = \sum_n M(A_n)$
	\end{itemize}

	Since $\forall A \in \mathscr{M}$, 

	$$\emptyset \subseteq A \subset \Omega $$

	then 

	$$ 0 \leq P^*(\emptyset) \leq P^*(A) \leq P^*(\Omega) \leq 1$$

	But, by Lemma 3, $P^*$ is contably additive on $\mathscr{M}$. So $P^*$ is probability measure on $\mathscr{M}$ (which is a $\sigma$-field, by Lemma 3).

	By Lemma 4, $\mathscr{F}_0 \subset \mathscr{M} \Rightarrow \sigma(\mathscr{F}_0 \subseteq \mathscr{M}$. So $P^*$ is also probabliity measure on $\sigma(\mathscr{F}_0).$

	Finally, by Lemma 5, again $P^*(A) = P(A)$, $P^*$ is an extention of $P$ form $\mathscr{F}_0$  to $\sigma(\mathscr{F}_0)$. 
\end{proof}

Uniqueness of of the extention, $\pi-\lambda$ Theorem.

	Paving - $\{ \pi$-system and $\lambda$-system. (?)

	A clan of subsets $\mathscr{P}$ of $\Omega$ is a $\pi$ system, if $A, B \in \mathscr{P} \Rightarrow AB \in \mathscr{P}$. 

	A class $\mathscr{L}$ is a $\lambda$-system if 
		\begin{enumerate}
			\item $\Omega \in \mathscr{L}$ 
			\item $A \in \mathscr{L} \Rightarrow A^C \in \mathscr{L}$
			\item If $A_1, dots \in \mathscr{L} $ are disjoint then $\cup^\infty_{n=1} A_n \in \mathscr{L}$
		\end{enumerate}

	So, the only difference is "disjoint". So, weaker than a $\sigma$-field (i.e. A $\sigma$-field is always a $\lambda$-system).

Note that $(\lambda_2)$ can be replace by $(\lambda_{2\prime})$ wherein

$$A, B \in \mathscr{F}, A \subseteq B, \Rightarrow B\setminus A \in \mathscr{L} $$

That is $\lambda_1, \lambda_2, \lambda_3 \Leftrightarrow \lambda_1, \lambda_{2\prime}, \lambda_3$

\textbf{Lemma 6.} A class of sets that is both $\pi$-systema and $\lambda$-system is a $\sigma$-field. 

\begin{proof}
	Suppose $\mathscr{F}$ is both $\pi$-systema and $\lambda$-system.

By definition, 
\begin{enumerate}
			\item $\Omega \in \mathscr{F}$ 
			\item $A \in \mathscr{F} \Rightarrow A^C \in \mathscr{F}$
		\end{enumerate}

		Let $A_1, A_2, \dots$ be $\mathscr{F}$ sets. 

		Let's constructs disjoints sets, B

		\begin{align*}
			B_1 &= A_1\\
			B_2 &= A_1A_2^C\\
			\vdots
		\end{align*}

		Then $B_n$ are $mathscr{F}$-sets (by $\lambda_{2\prime} - A_2^C = \Omega A)2^C \in \mathscr{F}$, by $\pi$-system, $A_1A_2^C \in \mathscr{F}$ ).\\

		By $\lambda_3$, 

		$$\cup_n^\infty B_n \in \mathscr{F} $$

		So, 

		$$\cup_n^\infty A_n \in \mathscr{F} $$

\end{proof}

\begin{theorem}[$\pi$-$\lambda$ Theorem]
	If $\mathscr{P}$ is in a $\pi$-system, $\mathscr{L}$ is in a $\lambda$-system, then 
	$$\mathscr{P} \subseteq \mathscr{L} \Rightarrow \sigma(\mathscr{P} \subseteq \mathscr{L}) $$
\end{theorem}

\begin{proof}
	Let $\lambda(\mathscr{P})$ be the intersection of all $\lambda$-system that contains $\mathscr{P}$. 

		$$ \lambda(\mathscr{P}) = \cap\{\mathscr{L}^\prime: \mathscr{L}^\prime \supseteq \mathscr{P}, \mathscr{L}^\prime \text{ is }\lambda\text{-set }\}$$

	$\lambda(\mathscr{P})$ is a $\lambda$-system.

	\begin{enumerate}
		\item $\Omega \in \lambda(\mathscr{P})$?\\

			$$\Omega \in \mathscr{L}^\prime \quad \forall \mathscr{L}^\prime$$
			$$\Omega \in \lambda(\mathscr{P}) $$

		\item $A \in \lambda(\mathscr{P}) \Rightarrow A^C \in \lambda(\mathscr{P})$?\\

		$$A \in \lambda(\mathscr{P}) \Rightarrow A \in \cap\{\mathscr{L}^\prime: \mathscr{L}^\prime \supseteq \mathscr{P}, \mathscr{L}^\prime \text{ is }\lambda\text{-set }\} $$

		Then 

		$A \in \mathscr{L}^\prime$ for any $\mathscr{L}^\prime \supseteq \mathscr{P}, \mathscr{L}^\prime$ is $\lambda$-system. 

		$$\Rightarrow A^C \in \mathscr{L}^\prime $$

		$$\Rightarrow A^C \in  \cap\{\mathscr{L}^\prime: \mathscr{L}^\prime \supseteq \mathscr{P}, \mathscr{L}^\prime \text{ is }\lambda\text{-set }\} = \lambda(\mathscr{P})$$

		\item $A_1, A_2, \dots \in \lambda(\mathscr{P})$ are disjoint then $A_1, A_2, \dots \in \mathscr{L}^\prime \quad \forall \mathscr{L}^\prime$. 

		Then $\cup A_n \in \mathscr{L}^\prime (mathscr{L}^\prime \lambda\text{-system})$

		So $\cup_n A_n \in \lambda(\mathscr{P})$. 

		We call $\lambda(\mathscr{P})$ the $\lambda$-system generated by $\mathscr{P}$.\\

		If we can say that $\lambda(\mathscr{P})$ is also a $\sigma$-field, then $\sigma(\mathscr{P}) \subseteq \lambda(\mathscr{P})$ because $\sigma(\mathscr{P})$ is smallest. So then, $\sigma(\mathscr{P}) \subseteq \mathscr{L}$ because $\lambda(\mathscr{P})$ is the small $\lambda$-system. 

		So it suffices to show that $\lambda(\mathscr{P})$ is a $\sigma$-field. But we know if $\lambda(\mathscr{P})$ is a system then $\lambda(\mathscr{P})$ is $\sigma$-field. So it suffices to show that $\lambda(\mathscr{P})$ is a $\pi$-system.  \\

		Construct again for any $A \in 2^\Omega \quad (A \subseteq \Omega)$, let

		$$\mathscr{L}_A = \{B: AB \in \lambda(\mathscr{P}) \}$$

		Claim: If $A \in \lambda(\mathscr{P})$ then $\mathscr{L}_A$ is $\lambda$-system.\\

		\begin{enumerate}
			\item $\Omega \in \mathscr{L}_A$? 
				$$A\Omega = A \in \mathscr{L}_A$$
			\item $(\lambda_2^\prime) : B_1, B_2 \in \mathscr{L}_A, B_1 \subseteq  B_2 \Rightarrow B_2B_1^C \in \mathscr{L}_A $?

			$$B_1 \in \mathscr{L}_A \Rightarrow AB_1 \in \lambda(\mathscr{P}) $$
			$$B_2 \in \mathscr{L}_A \Rightarrow AB_2 \in \lambda(\mathscr{P}) $$

			Since $AB_1 \subseteq AB_2$, $\lambda(\mathscr{P})$ is $\lambda$-system by ($\lambda_2^\prime$) for $\lambda(\mathscr{P})$

			% GET FROM PHOTO

			\item If $B_n$ is disjoint, $\mathscr{L}_A$-sets. 

			Want $\cup_n B_n$ because 

			$$B_n \in \mathscr{L}_A $$

			$$B_n A \in \lambda(\mathscr{P}) $$

			Because $B_n$ disjoint we know that $B_n A$ is also disjoint.

			Hence, 
			$$\cup_n(B_n A) \in \lambda(\mathscr{P})$$


		\end{enumerate}
	\end{enumerate}
\end{proof}

% %------------------------------------------------

% \section{Citation}\index{Citation}

% This statement requires citation \cite{book_key}; this one is more specific \cite[122]{article_key}.

% %------------------------------------------------

% \section{Lists}\index{Lists}

% Lists are useful to present information in a concise and/or ordered way\footnote{Footnote example...}.

% \subsection{Numbered List}\index{Lists!Numbered List}

% \begin{enumerate}
% \item The first item
% \item The second item
% \item The third item
% \end{enumerate}

% \subsection{Bullet Points}\index{Lists!Bullet Points}

% \begin{itemize}
% \item The first item
% \item The second item
% \item The third item
% \end{itemize}

% \subsection{Descriptions and Definitions}\index{Lists!Descriptions and Definitions}

% \begin{description}
% \item[Name] Description
% \item[Word] Definition
% \item[Comment] Elaboration
% \end{description}

%----------------------------------------------------------------------------------------
%	CHAPTER 2
%----------------------------------------------------------------------------------------

\chapter{General Measure}

% \section{Theorems}\index{Theorems}

% This is an example of theorems.

% \subsection{Several equations}\index{Theorems!Several Equations}
% This is a theorem consisting of several equations.

% \begin{theorem}[Name of the theorem]
% In $E=\mathbb{R}^n$ all norms are equivalent. It has the properties:
% \begin{align}
% & \big| ||\mathbf{x}|| - ||\mathbf{y}|| \big|\leq || \mathbf{x}- \mathbf{y}||\\
% &  ||\sum_{i=1}^n\mathbf{x}_i||\leq \sum_{i=1}^n||\mathbf{x}_i||\quad\text{where $n$ is a finite integer}
% \end{align}
% \end{theorem}

% \subsection{Single Line}\index{Theorems!Single Line}
% This is a theorem consisting of just one line.

% \begin{theorem}
% A set $\mathcal{D}(G)$ in dense in $L^2(G)$, $|\cdot|_0$. 
% \end{theorem}

% %------------------------------------------------

% \section{Definitions}\index{Definitions}

% This is an example of a definition. A definition could be mathematical or it could define a concept.

% \begin{definition}[Definition name]
% Given a vector space $E$, a norm on $E$ is an application, denoted $||\cdot||$, $E$ in $\mathbb{R}^+=[0,+\infty[$ such that:
% \begin{align}
% & ||\mathbf{x}||=0\ \Rightarrow\ \mathbf{x}=\mathbf{0}\\
% & ||\lambda \mathbf{x}||=|\lambda|\cdot ||\mathbf{x}||\\
% & ||\mathbf{x}+\mathbf{y}||\leq ||\mathbf{x}||+||\mathbf{y}||
% \end{align}
% \end{definition}

% %------------------------------------------------

% \section{Notations}\index{Notations}

% \begin{notation}
% Given an open subset $G$ of $\mathbb{R}^n$, the set of functions $\varphi$ are:
% \begin{enumerate}
% \item Bounded support $G$;
% \item Infinitely differentiable;
% \end{enumerate}
% a vector space is denoted by $\mathcal{D}(G)$. 
% \end{notation}

% %------------------------------------------------

% \section{Remarks}\index{Remarks}

% This is an example of a remark.

% \begin{remark}
% The concepts presented here are now in conventional employment in mathematics. Vector spaces are taken over the field $\mathbb{K}=\mathbb{R}$, however, established properties are easily extended to $\mathbb{K}=\mathbb{C}$.
% \end{remark}

% %------------------------------------------------

% \section{Corollaries}\index{Corollaries}

% This is an example of a corollary.

% \begin{corollary}[Corollary name]
% The concepts presented here are now in conventional employment in mathematics. Vector spaces are taken over the field $\mathbb{K}=\mathbb{R}$, however, established properties are easily extended to $\mathbb{K}=\mathbb{C}$.
% \end{corollary}

% %------------------------------------------------

% \section{Propositions}\index{Propositions}

% This is an example of propositions.

% \subsection{Several equations}\index{Propositions!Several Equations}

% \begin{proposition}[Proposition name]
% It has the properties:
% \begin{align}
% & \big| ||\mathbf{x}|| - ||\mathbf{y}|| \big|\leq || \mathbf{x}- \mathbf{y}||\\
% &  ||\sum_{i=1}^n\mathbf{x}_i||\leq \sum_{i=1}^n||\mathbf{x}_i||\quad\text{where $n$ is a finite integer}
% \end{align}
% \end{proposition}

% \subsection{Single Line}\index{Propositions!Single Line}

% \begin{proposition} 
% Let $f,g\in L^2(G)$; if $\forall \varphi\in\mathcal{D}(G)$, $(f,\varphi)_0=(g,\varphi)_0$ then $f = g$. 
% \end{proposition}

% %------------------------------------------------

% \section{Examples}\index{Examples}

% This is an example of examples.

% \subsection{Equation and Text}\index{Examples!Equation and Text}

% \begin{example}
% Let $G=\{x\in\mathbb{R}^2:|x|<3\}$ and denoted by: $x^0=(1,1)$; consider the function:
% \begin{equation}
% f(x)=\left\{\begin{aligned} & \mathrm{e}^{|x|} & & \text{si $|x-x^0|\leq 1/2$}\\
% & 0 & & \text{si $|x-x^0|> 1/2$}\end{aligned}\right.
% \end{equation}
% The function $f$ has bounded support, we can take $A=\{x\in\mathbb{R}^2:|x-x^0|\leq 1/2+\epsilon\}$ for all $\epsilon\in\intoo{0}{5/2-\sqrt{2}}$.
% \end{example}

% \subsection{Paragraph of Text}\index{Examples!Paragraph of Text}

% \begin{example}[Example name]
% \lipsum[2]
% \end{example}

% %------------------------------------------------

% \section{Exercises}\index{Exercises}

% This is an example of an exercise.

% \begin{exercise}
% This is a good place to ask a question to test learning progress or further cement ideas into students' minds.
% \end{exercise}

% %------------------------------------------------

% \section{Problems}\index{Problems}

% \begin{problem}
% What is the average airspeed velocity of an unladen swallow?
% \end{problem}

% %------------------------------------------------

% \section{Vocabulary}\index{Vocabulary}

% Define a word to improve a students' vocabulary.

% \begin{vocabulary}[Word]
% Definition of word.
% \end{vocabulary}

%----------------------------------------------------------------------------------------
%	PART
%----------------------------------------------------------------------------------------

% \part{Part Two}

%----------------------------------------------------------------------------------------
%	CHAPTER 3
%----------------------------------------------------------------------------------------

\chapterimage{chapter_head_1.pdf} % Chapter heading image

\chapter{Integration with Respect to a Measure}



% \section{Table}\index{Table}

% \begin{table}[h]
% \centering
% \begin{tabular}{l l l}
% \toprule
% \textbf{Treatments} & \textbf{Response 1} & \textbf{Response 2}\\
% \midrule
% Treatment 1 & 0.0003262 & 0.562 \\
% Treatment 2 & 0.0015681 & 0.910 \\
% Treatment 3 & 0.0009271 & 0.296 \\
% \bottomrule
% \end{tabular}
% \caption{Table caption}
% \end{table}

% %------------------------------------------------

% \section{Figure}\index{Figure}

% \begin{figure}[h]
% \centering\includegraphics[scale=0.5]{placeholder}
% \caption{Figure caption}
% \end{figure}


%----------------------------------------------------------------------------------------
%	CHAPTER 4
%----------------------------------------------------------------------------------------

\chapterimage{chapter_head_1.pdf} % Chapter heading image

\chapter{Random Variable}


%----------------------------------------------------------------------------------------
%	CHAPTER 5
%----------------------------------------------------------------------------------------

\chapterimage{chapter_head_1.pdf} % Chapter heading image

\chapter{Convergence in Probability/Limit Theorem}


%----------------------------------------------------------------------------------------
%	CHAPTER 6
%----------------------------------------------------------------------------------------

\chapterimage{chapter_head_1.pdf} % Chapter heading image

\chapter{Radon-Nikodym Derivative Theorem}

%----------------------------------------------------------------------------------------
%	CHAPTER 7
%----------------------------------------------------------------------------------------

\chapterimage{chapter_head_1.pdf} % Chapter heading image

\chapter{Special Topics}


%----------------------------------------------------------------------------------------
%	BIBLIOGRAPHY
%----------------------------------------------------------------------------------------

% \chapter*{Bibliography}
% \addcontentsline{toc}{chapter}{\textcolor{ocre}{Bibliography}}
% \section*{Books}
% \addcontentsline{toc}{section}{Books}
% \printbibliography[heading=bibempty,type=book]
% \section*{Articles}
% \addcontentsline{toc}{section}{Articles}
% \printbibliography[heading=bibempty,type=article]

%----------------------------------------------------------------------------------------
%	INDEX
%----------------------------------------------------------------------------------------

\cleardoublepage
\phantomsection
\setlength{\columnsep}{0.75cm}
\addcontentsline{toc}{chapter}{\textcolor{ocre}{Index}}
\printindex

%----------------------------------------------------------------------------------------

\end{document}