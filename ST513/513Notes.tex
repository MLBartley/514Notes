%%%%%%%%%%%%%%%%%%%%%%%%%%%%%%%%%%%%%%%%%
% The Legrand Orange Book
% LaTeX Template
% Version 2.0 (9/2/15)
%
% This template has been downloaded from:
% http://www.LaTeXTemplates.com
%
% Mathias Legrand (legrand.mathias@gmail.com) with modifications by:
% Vel (vel@latextemplates.com)
%
% License:
% CC BY-NC-SA 3.0 (http://creativecommons.org/licenses/by-nc-sa/3.0/)
%
% Compiling this template:
% This template uses biber for its bibliography and makeindex for its index.
% When you first open the template, compile it from the command line with the 
% commands below to make sure your LaTeX distribution is configured correctly:
%
% 1) pdflatex main
% 2) makeindex main.idx -s StyleInd.ist
% 3) biber main
% 4) pdflatex main x 2
%
% After this, when you wish to update the bibliography/index use the appropriate
% command above and make sure to compile with pdflatex several times 
% afterwards to propagate your changes to the document.
%
% This template also uses a number of packages which may need to be
% updated to the newest versions for the template to compile. It is strongly
% recommended you update your LaTeX distribution if you have any
% compilation errors.
%
% Important note:
% Chapter heading images should have a 2:1 width:height ratio,
% e.g. 920px width and 460px height.
%
%%%%%%%%%%%%%%%%%%%%%%%%%%%%%%%%%%%%%%%%%

%----------------------------------------------------------------------------------------
%	PACKAGES AND OTHER DOCUMENT CONFIGURATIONS
%----------------------------------------------------------------------------------------

\documentclass[11pt,fleqn]{book} % Default font size and left-justified equations

%----------------------------------------------------------------------------------------

\input{structure} % Insert the commands.tex file which contains the majority of the structure behind the template

\begin{document}

%----------------------------------------------------------------------------------------
%	TITLE PAGE
%----------------------------------------------------------------------------------------

\begingroup
\thispagestyle{empty}
\begin{tikzpicture}[remember picture,overlay]
\coordinate [below=12cm] (midpoint) at (current page.north);
\node at (current page.north west)
{\begin{tikzpicture}[remember picture,overlay]
\node[anchor=north west,inner sep=0pt] at (0,0) {\includegraphics[width=\paperwidth]{background}}; % Background image
\draw[anchor=north] (midpoint) node [fill=ocre!30!white,fill opacity=0.6,text opacity=1,inner sep=1cm]{\Huge\centering\bfseries\sffamily\parbox[c][][t]{\paperwidth}{\centering Theory of Statistics I\\[15pt] % Book title
{\Large Take Two}\\[20pt] % Subtitle
{\huge Meridith L Bartley}}}; % Author name
\end{tikzpicture}};
\end{tikzpicture}
\vfill
\endgroup

%----------------------------------------------------------------------------------------
%	COPYRIGHT PAGE
%----------------------------------------------------------------------------------------

\newpage
~\vfill
\thispagestyle{empty}

\noindent Copyright \copyright\ 2013 Meridith L Bartley\\ % Copyright notice

\noindent \textsc{Published by Publisher}\\ % Publisher

\noindent \textsc{book-website.com}\\ % URL

\noindent Licensed under the Creative Commons Attribution-NonCommercial 3.0 Unported License (the ``License''). You may not use this file except in compliance with the License. You may obtain a copy of the License at \url{http://creativecommons.org/licenses/by-nc/3.0}. Unless required by applicable law or agreed to in writing, software distributed under the License is distributed on an \textsc{``as is'' basis, without warranties or conditions of any kind}, either express or implied. See the License for the specific language governing permissions and limitations under the License.\\ % License information

\noindent \textit{First printing, August 2015} % Printing/edition date

%----------------------------------------------------------------------------------------
%	TABLE OF CONTENTS
%----------------------------------------------------------------------------------------

\chapterimage{chapter_head_1.pdf} % Table of contents heading image

\pagestyle{empty} % No headers

\tableofcontents % Print the table of contents itself

\cleardoublepage % Forces the first chapter to start on an odd page so it's on the right

\pagestyle{fancy} % Print headers again

%----------------------------------------------------------------------------------------
%	PART
%----------------------------------------------------------------------------------------

\part{Part One}

%----------------------------------------------------------------------------------------
%	CHAPTER 1
%----------------------------------------------------------------------------------------

\chapterimage{chapter_head_2.pdf} % Chapter heading image

\chapter{Real Analysis Review}

\section{The Real Number System}\index{The Real Number System}
\subsection{Rationals}\index{Rationals}


Start with integers as given. 

\begin{definition}[Rational Numbers]
Rationals are numbers of the form $\frac{m}{n}$, for m,n integers, $n \neq 0$ such that:
\begin{align*}
& \text{PR 1: The sum, difference, product, and ratio (division by 0 excluded) of any two rationals is a rational.}\\
& \text{PR 2: } p+q = q+p, pq=qp \text{ (Commutative Property)}\\
& \text{PR 3: } (p+q)+r = p+(q+r), (pq)r = p(qr), \text{ (Associative Property)} \\
& \text{PR 4: } (p+q)r = pr + qr \text{ (Distributive Property)} \\
& \text{PR 5: } \forall \text{ two rationals p and q we have either p=q, p<q, or q<p (Ordering Property)} \\
& \text{PR 6: If $p<q$ and $q<r$, then $p<r$ (Transitivity of <)}\\
& \text{PR 7: If $p>0$ and $q>0$, then $p+q>0$ and $pq >0$} \\
& \text{PR 8: If $p<q$, then $p+r<q+r$  $\forall$ r} \\
\end{align*}
\end{definition}

The rational number system is inadequate. 

\begin{example}
	There is no rational number p that satisfies $p^2 = 2$
\end{example}

	\begin{proof}
		Suppose such a p existed, and so $p=\frac{m}{n}$. Note that m,n can be chosen so not both are even. Then we have,
		$$m^2 = 2n^2 $$
		Thus, $m^2$ is even, and hence m is even. (The square of an odd number is odd). Hence, $m^2$ is divided by 4. So, $2n^2$ is divisible by 4, or $n^2$ is even which implies that n is even - $\textbf{contradiction}$.
	\end{proof}

This example can be used to show that we can have a set of rational numbers bounded from above but has no supremum. 

\begin{example}
	Let A be the set of $<0$ rationals p, such that $p^2 <2$. Let B be the set of $>0$ rationals p, such that $p^2 >2$. Then A contains no largest number and B contains no smallest number.
\end{example}

\begin{proof}
	If $p \in A$, choose a rational h such that, $0<h<1$ and $h<\frac{2-p^2}{2p+1}$ and set $q=p+h$. Then q is rational and 
	\begin{align*}
		q^2 &= p^2 + (2p+h)h\\
		&< p^2 + (2p+1)h\\
		&< p^2 + (2-p^2)\\
		&= 2
	\end{align*}
	If $p \in B$, set
	
	\begin{align*}
			 q= p- \frac{p^2 - 2}{2p} = \frac{p}{2} + \frac{1}{p}
	\end{align*}
	and
	\begin{align*}
		q^2 &= p^2 - (p^2 - 2) + (\frac{p^2 - 2}{2p})^2\\
		&> p^2 - (p^2 -2)\\
		&= 2
	\end{align*}
\end{proof}

\begin{remark}
	An axiomatic treatment of the real number system uses PR1 - PR8 as axioms together with the "completeness axiom". The non-axiomatics treatment is due to Dedekind.
\end{remark}


\subsection{Sets and Subsets}\index{Sets and Subsets}

If A is any set, $\mathbf{x \in A}$ means that x is a member of A, and $\mathbf{x \notin A}$ means x is not a member of A. A set B is a \textbf{subset} of A if for every $x \in B$ we have $x \in A$, and we write $A \subseteq B$. B is a \textbf{proper subset} of A, $B \subset A$, if there $\exists$ $x \in A$ with $x \notin B$. The \textbf{empty set} is denoted by $\emptyset$, and $\emptyset \in A$, $\forall$ other set A.

\begin{align*}
	A \cup B &= B \cup A \text{ - union with commutative property}\\
	\\
	A \cap B &= B \cap A \text{ - intersection with commutative property}\\
	\\
	(A \cup B)\cup C &= A\cup(B \cup C)\\
	(A \cap B)\cap C &= A\cap(B \cap C) \text{ - associative property}\\
	\\
	 (A \cup B)\cap C &= (A\cap C)\cup(B \cap C) \text{ - distributive property}\\
	 (\cup A_i)^c &= \cap A_i^c\\
	 (\cap A_i)^c &= \cup A_i^c\\
\end{align*}

\begin{definition}[Dedekind Cuts]
	A set $\alpha$ of rational numbers is said to be a \textbf{cut} if
	\begin{enumerate}
		\item $\alpha$ is a proper, but non-empty, subset of the rational numbers.
		\item If $p \in \alpha$ (p is rational), and $q<p$ (q is rational) then $q \in \alpha$
		\item It contains no largest rational.
	\end{enumerate}
	A cut of the form $\alpha$ = $\{$p: p is rational and $p<r\}$ where r is rational are called \textbf{rational cuts} and are denoted by $r^*$.
\end{definition}

The development of the real number system proceeds as follows:\\
First, the set of cuts is equipped with an order relation, and the operation of addition and multiplication an it will show that the resulting arethmatic satisfies PR 1 - PR 8.  

If $\alpha$, $\beta$ are cuts then, 
\begin{align*}
	\alpha &< \beta \text{ if $\alpha \subset \beta$ and} \\
	\alpha &\le \beta \text{ if $\alpha \subseteq \beta$}\\
	\alpha + \beta &= \{r: r= p + q \text{ for some } p \in \alpha, q \in \beta\}\\
	(\alpha + 0^* &= \alpha)
\end{align*}

If $\alpha + \beta = 0^*$, write $\beta = -\alpha$. (It can be shown that $\forall \alpha$ there is one and only one $\beta$ such that $\alpha + \beta = 0^*$.) 
$$|\alpha| = \begin{cases}
  \alpha, & \text{if } \alpha \ge 0^*, \\
  -\alpha, & \text{if } \alpha < 0^* .
\end{cases} $$

For $\alpha \ge 0^*$ and $\beta \ge 0^*$, 
$$\alpha\beta = \{\text{p:p rational such that either $p<0$ or $p=pq$, for $q \in \alpha$, $r \in \beta$ with $q \ge 0$ and $r \ge 0$.}\}$$ 

For general $\alpha$, $\beta$,
$$\alpha\beta = \begin{cases}
  -(|\alpha||\beta|), & \text{if } \alpha < 0^*, \text{and } \beta \ge 0^*\\
  	& \text{or if } \alpha \ge 0^* \text{and } \beta < 0^*\\
  |\alpha||\beta|, & \text{if } \alpha < 0^*, \text{and } \beta < 0^*\\
\end{cases} $$

If $\alpha \neq 0^*$, then $\forall \beta$ there is one and only one $\gamma$ such that $\alpha\gamma = \beta$, and this $\gamma$ is denoted by $\frac{\beta}{\alpha}$. (In technical terms, we made the set of cuts an \textbf{ordered field}.)

Second, it is shown that replacing the set of rational numbers by the corresponding cuts preserves sums, products and order, ie,

\begin{enumerate}
	\item $p^* + q^* =(p + q)^*$ 
	\item $p^*q^* = (pq)^*$
	\item $p^* < q^*$ iff $p<q$
\end{enumerate}
In technical terms, the ordered field of rational numbers is \textbf{isomorphic} to that of rational events.

\begin{theorem}[Dedekind]
	Let A, B be $\subset \mathbb{R}$ such that,
	\begin{enumerate}[label = (\alph*)]
		\item $A \cap B = \emptyset$
		\item $A \cup B = \mathbb{R}$
		\item neither $A$ nor $B$ is empty
		\item if $\alpha \in A$, $\beta \in B$, then $\alpha < \beta$
	\end{enumerate}
	Then there $\exists$ $\gamma \in \mathbb{R}$ such that $\alpha \le \gamma$, $\forall \alpha \in A$ and $\gamma \le \beta$, $\forall \beta \in B$. 
\end{theorem}

\begin{proof}
	First, suppose there are 2 $\gamma$, say $\gamma_1 < \gamma_2$. Take $\gamma_3$ such that $\gamma_1 < \gamma_3 < \gamma_2$.
	$$\gamma_3 < \gamma_2 \text{ implies that } \gamma_3 \in A$$
	$$\gamma_1 < \gamma_3 \text{ implies that } \gamma_3 \in B$$
However, these implications contradict the disjointness (part (a)). 
Define $\gamma - \{\text{p: p rational such that p }\in \alpha \text{ for some }\alpha \in A\}$. The proof proceeds by showing that $\gamma$ is a cut, and hense a real number that satisfies $\alpha \le \gamma$ for $\alpha \in A$ and $\gamma \le \beta$ $\forall$ $\beta \in B$.
\end{proof}

\begin{corollary}
	If A, B are as in the theorem, then either A contains a largest number or B contains a smallest number. 
\end{corollary}

\begin{corollary}	
	Let $E \neq \emptyset$ be a subset of $\mathbb{R}$. Then, if E is bounded above a supremum (least upper bound) exists.
\end{corollary}	

\begin{proof}
	Define \\
	$A = \{\alpha: \alpha < x \text{ for some }x \in E\}$\\
	$B = A^c$\\
	Clearly, all members of B are upper bounds of E. It is sufficient to prove that B contains a smallest nubmer, or, by Corollary 1, that A does not contain a largest number (and thus prove by contradiction). Indeed if $\alpha \in A$ $\exists$ an $x \in E$ such that $\alpha < x$. But, by Property 1 (???) there $\exists$ an $\alpha^\prime$ such that $\alpha < \alpha^\prime < x$ where $\alpha^\prime \in A $ (i.e. we can always find a larger $\alpha$ so, since there is no largest $\alpha$, there MUST be a smallest $\beta$).  
\end{proof}

\begin{theorem}
	Any real number admits a decimal expansion.
\end{theorem}

\begin{proof}
	Let $x > 0$, $x \in \mathbb{R}$. Let $n_0 = [x]$ (n largest integer < x). Let $n_1$ be the largest integer such that $n_0 + \frac{n_1}{10} < x$. Having defined $n_0 \dots n_{k-1}$, define $n_k$ as the largest integer such that $n_0 + \frac{n_1}{10} + \frac{n_2}{10^2} + \dots + \frac{n_k}{1-^k} \le x$. Let E bet he set of resluting numbers for $k=1,2,\dots$. Then $x$ is the supremum of E and $n_0, n_1, \dots$ is its \textbf{decimal expansion}. Conversely, and set of integers $n_0, n_1, \dots$ defines a set of numbers, E, bounded above by $n_0 + 1$.	
\end{proof}

\begin{definition}[Extended Real Number System]
	$$\bar{\mathbb{R}} = \mathbb{R} \cup \{-\infty, \infty\}$$
\end{definition}

\subsection{Euclidean Space}\index{Euclidean Space}

\begin{definition}[Vector Space]
	For any $k \in \mathbb{Z}^+$. Let $\mathbb{R}^K$ be the set of ordered k-tuples. 
	\begin{align*}
	\underline{x} &= (x_1, \dots, x_k) \text{ with }x_i \in \mathbb{R}\\
	\underline{y} &= (y_1, \dots, y_k) \alpha \in \mathbb{R}\\
	\underline{x} + \underline{y} &= (x_1 + y_1, \dots, x_k + y_k)\\
	\alpha\underline{x} &	= (\alpha x_1, \dots, \alpha x_k)\\
	\end{align*}
	Which makes $\mathbb{R}^k$ a \textbf{vector space} over the \textbf{real field}.
\end{definition}

\begin{definition}[Inner/Scalar/Dot Product]
$$ \underline{x} \cdot \underline{y} = \sum\limits_{i=1}^k x_i y_i$$	
\end{definition}

\begin{definition}[Norm/Length]
	$$|\underline{x}| = (\underline{x}\underline{x})^{\frac{1}{2}} = \sqrt{\sum\limits_{i=1}^k x_i^2} $$
\end{definition}

\begin{definition}[Euclidean K-space]
	The vector space $\mathbb{R}^k$ with the inner product and norm is called \textbf{Euclidean k-space}.
\end{definition}

\begin{theorem}
	For $\underline{x}, \underline{y} \in \mathbb{R}^k, \alpha \in \mathbb{R}$
	\begin{enumerate}[label = \alph*)]
		\item $|\underline{x}| \ge 0, |\underline{x}| = 0 \text{ iff } \underline{x} = \underline{0}$\\
		$|\alpha \underline{x}| = |\underline{\alpha}||\underline{x}|$ 
		\item \textbf{Cauchy-Schwarz Inequality} $|\underline{x}\cdot\underline{y}| \le |\underline{x}||\underline{y}|$
		\item \textbf{Triangle Inequality} $|\underline{x}+\underline{y}| \le |\underline{x}| + |\underline{y}|$
	\end{enumerate}
\end{theorem}

\section{Elements of Set Theory}\index{Elements of Set Theory}

\begin{definition}
	Let A, B be sets and suppose that to each $x \in A$ there corresponds an elements of B denoted by $f(x)$. Then $f$ is a \textbf{function} (or in more general space, mapping) from A (in)to B. 

	A is called the \textbf{domain} of $f$. $f(x)$ is the \textbf{value} of $f$ at $x$, $R(f) = \{f(x): x \in A\}$ is the \textbf{range} of $f$.
\end{definition}

\begin{definition}[Image]
	If $f$ is a function from A to B ($A \rightarrow B$) and $E \subseteq A$ we write $f(E) = \{f(x): x \in E\}$ and call it the \textbf{image} of E under $f$.
	If $f(A) = B$, then we say $f$ maps A \textbf{onto} B.
\end{definition}

\begin{definition}[Inverse Image]
	Let $f: A \rightarrow B$ and $E \subseteq B$. We write $f^{-1}(E) = \{x ]in A : f(x) \in E\}$ and call it the \textbf{inverse image} of E \textbf{under} $f$.
	NB: If $E =\{y\}, y \in B$ we also write $f^{-1}(y)$ (versus $f^{-1}(\{y\})$). If $\forall y \in B$ $f^{-1}(y)$ consists of at most one element, then $f$ is one to one mapping of A \textbf{into} B.
\end{definition}

\begin{theorem}  
	\begin{enumerate}[label = \alph*)]
		\item $f^{-1}(A\cup B) = f^{-1}(A)\cup f^{-1}(B)$
		\item $f(A\cup B) = f(A) \cup f(B)$
		\item $f^{-1}(A\cap B) = f^{-1}(A) \cap f^{-1}(B)$
	\end{enumerate}
	Actually, these may be extended to arbitrary unions and intersections.
	\begin{align*}	
		f^{-1}(\bigcup\limits_\alpha A_\alpha) &= \bigcup_\alpha f^{-1}(A_\alpha)\\
		f^{-1} (\bigcap_\alpha A_\alpha) &= \bigcap_\alpha f^{-1}(A_\alpha)\\
		f(\bigcup_\alpha A_\alpha) &= \bigcup_\alpha f(A_\alpha)
	\end{align*}	
	Note: $f(A\cap B)$ is not necessarily equal to $f(A) \cap f(B)$ (see notes for example and sketch)
\end{theorem}

\begin{definition}[Cardinal Number]
	If $\exists$ a one-to-one mapping of A onto B, we say that A and B have the same \textbf{cardinal number}, or that they are \textbf{equivalent} $A\sim B$.  
	\begin{enumerate}[label = \alph*)]
		\item $A \sim A$ (reflective)
		\item If $A \sim B$, then $B \sim A$ (symmetric)
		\item If $A \sim B$ and $B \sim C$, then $A \sim C$. (transitive)
	\end{enumerate}
\end{definition}

\begin{definition}[(In)finite/(Un)Countable]
	Let $\mathbb{Z}^+ = \{1, 2, \dots \}$ and $\mathbb{Z}_n^+ = \{1, 2, \dots, n\}$ and let A be a set.
	\begin{enumerate}[label = \alph*)]
		\item We say A is \textbf{finite} if $A \sim \mathbb{Z}^+_n$ for some n or if $A = \emptyset$
		\item A is \textbf{infinite} if it is not finite
		\item A is \textbf{countable} if $A \sim \mathbb{Z}^+$
		\item A is \textbf{uncountable} if A is not finite and countable.
	\end{enumerate}
	Note: If A and B are finite, then $A \sim B$ if and only if they have the same number of elements. This is not true if they are infinite. 
\end{definition}

\begin{example}
	\textbf{Equivalent Infinite Sets}
	\begin{enumerate}
		\item The set $\mathbb{Z}^+$ of all integers is countable. Then take
			 $$ f(x) = \begin{cases}
	  		\frac{n}{2}, & \text{if n is even} \\
	  		-(\frac{n-1}{2}), & \text{if n is odd} 
			\end{cases} $$
			\end{enumerate}

			\begin{table}[h]
			\centering
			\begin{tabular}{l l l}
			1 & $\rightarrow$ & 0 \\
			2 & $\rightarrow$ & 1 \\
			3 & $\rightarrow$ & -1 \\
			4 & $\rightarrow$ & 2 \\
			5 & $\rightarrow$ & -2 \\
			6 & $\rightarrow$ & 3 \\
			7 & $\rightarrow$ & -3 \\
			\end{tabular}
			\caption{Corresponding Integers}
			\end{table}
		\item The set of positive, even integers is countable. Take
	$$f(x)=2n $$	
\end{example}

\begin{theorem}
	The countable union of countable sets is countable.
\end{theorem}

\begin{proof}
	Let $A_1, A_2, \dots$ be countable and assume that they are disjoint (for if not, you can consider the sequences of countable sets that are disjoint - $A_1, A_2-A_1, \dots$), which are countable and have the same union. 
	Let $A_k = \{a_{k1}, a_{k2}, \dots\}$ and consider the arrangement of $\bigcup\limits_{k=1}^\infty A_k$.
		
		\begin{table}[h]
			\centering
			\begin{tabular}{l l l l l l l l l l l l}
			$a_{11}$ & $a_{12}$ & $a_{13}$ & $a_{14}$ & $\dots$ & 	 & 	 & 1 & 2 & 6 & 7 & $\dots$  \\
			$a_{21}$ & $a_{22}$ & $a_{23}$ & $a_{24}$ & $\dots$ & 	 & 	 & 3 & 5 & 8 & $\dots $ & $\dots$  \\
			$a_{31}$ & $a_{32}$ & $a_{33}$ & $a_{34}$ & $\dots$ & 	 & 	 & 4 & 9 & 13 & $\dots $ & $\dots$  \\
			$a_{41}$ & $a_{42}$ & $a_{43}$ & $a_{44}$ & $\dots$ & 	 & 	 & 10 & 12 & $\dots$ & $\dots $ & $\dots$  \\
			\end{tabular}
			\caption{Reassigning new values to counting integers.}
			\end{table}

\end{proof}

\begin{theorem}
	Every infinite set has a countable subset.
\end{theorem}

\begin{proof}
	Let $a_1$ be any element of A. Since A is infinite, it contains an $a_2 \neq a_1 \dots$. So it contains a countable subset.
\end{proof}

\begin{theorem}
	Every infinite set, A, is equivalent to at least one of its proper subsets.
\end{theorem}

\begin{proof}
	Let $E - \{a_1, a_2,\dots \}$ be a countable subset of A (which exists by previous Theorem).\\
	Write, 
	\begin{align*}
		E &= E_1 \cup E_2\\
		E_1 &= \{a_{odd}\}\\
		E_2 &= \{a_{even}\}\\
	\end{align*}

	Then, $E \sim E_2$

	Define,
	\begin{align*}
		&g: E \rightarrow E_2\\
		&g(a_i) = a_{2i}\\
	\end{align*}
		$f(a) = \begin{cases}
				  	a, & \text{if } a \notin E, \\
				  	g(a), & \text{if } a \in E.
					\end{cases} $\\
	So, we can also say that $A- E_1 \subset A$ and thus, $A \sim (A-E_1)$
\end{proof}

\begin{theorem}
	The set of real numbers in [0,1] is uncountable.
\end{theorem}

\begin{proof}
	Suppose all numbers in [0,1] are countable, $\{a_1, a_2, \dots \}$.

	Write them in decimal expansion form. So, we can say
	\begin{align*}
		a_1 &= 0.a_{11} a_{12}\dots a_{1n}\dots\\
		a_2 &= 0.a_{21} a_{22}\dots a_{2n}\dots\\
	\end{align*}

	Recall,
	\begin{align*}
	0 &= 0.000000000\dots\\
	1 &= 0.999999999\dots\\	
	\end{align*}
	
	Now, consider the number, $\beta$ with decimal expansion $\beta = 0.b_1b_2\dots$ where \\
	$$b_n = \begin{cases}
	1, & \text{if } a_{nn} = 1, \\
	2, & \text{if } a_{nn} \neq 1.
	\end{cases}$$
	There will always be 1 element difference. ALWAYS.
\end{proof}

\begin{theorem}
	If A is countable, then so is $A^n$, where 

	$$A^n = \{ (a_1, \dots, a_n); a_i \in A\} $$
\end{theorem}

\begin{proof}
	Statement is true for any n=1 since $A^1 = A$. Assume true for n=k. To show $A^{k+1}$ is countable, write an element ($a_1, a_2, \dots, a_k, a_{k+1}) = (\underline{a}, a_{k+1}), \underline{a} \in A^k$. Thus, $A^{k+1} = \bigcup\limits_{\underline{a}\in A^k} \{\underline{a}, a_{k+1}); a_k \in A\} $ (see previous Theorem).
\end{proof}

\subsection{Metric Spaces}\index{Metric Spaces} % (fold)

\begin{definition}
	A set X is a \textbf{metric space} is $\forall x, x \in X$ there is a \textbf{real} number, $d(x_1, x_2)$  called the \textbf{distance} between $x_1$ and $x_2$ such that,

	\begin{enumerate}[label = \alph*)]
		\item $d(x_1, x_2) > 0$ if $x_1 \neq x_2$ and $d(x_1, x_1) = 0$
		\item $d(x_1, x_2) = d(x_2, x_1)$
		\item $d(x_1, x_2) \le d(x_1, x_3) + d(x_2, x_3), \forall x_3$
	\end{enumerate}

\end{definition}

\begin{example}
	\begin{enumerate}[label = \alph*)]
		\item Euclidean spaces $\mathbb{R}^k$ are metric spaces with $d(x_1, x_2) = |x_1 - x_2|$
		\item Any subset of a metric space is a metric space with same distance.
		\item The set $\mathbb{R}^k$ can also be metrized with\\
			$d_1(x_1, x_2) = \sum\limits_{i=1}^k |x_{1i}-x_{2i} |$\\
			or with\\
			$d_2(x_1, x_2) = (\sum\limits_{i=1}^k |x_{1i}-x_{2i} |^p)^{\frac{1}{p}}$
		\item The set $C_{[a,b]}$ of all continuous functions on $[a,b]$ with\\
			$d_1(f,g) = \max\limits_{a \le t \le b} |f(t) - g(t)| $\\
			or with \\
			$d_2(f,g)= (\int\limits_a^b [f(t)-g(t)]^2)^{\frac{1}{2}} $
		\item The set $l_p$ of all infinite sequences $x= (x_1 x_2, \dots)$ satisfying $\sum\limits_{i=1}^\infty|x_i|^p < \infty$ for $p \ge 1$ with\\
		$d(x_1, x_2) = \sum\limits_{i=1}^\infty |x_{1i}-x_{2i} |^p)^{\frac{1}{p}}$

	\end{enumerate}
\end{example}

\begin{definition}
	Let X be a metric space. All sets and points mentioned are sets and elements of X.

	\begin{enumerate}[label = \alph*)]
		\item An \textbf{open ball} of radius r and center x is
			$$B(x,r) = \{y \in X: d(x,y) <r\} $$
			The \textbf{closed ball} is
			$$B[x,r] = \{y \in X: d(x,y) \le r\}  $$
			Open ball with center x are also called \textbf{neighborhoods} of x and $B(x,y)$ is denoted by $N_r(x)$.
		\item A point x is a \textbf{limit point} of a set E if $\forall r >0$ $E \cap N_r(x) $ 	contains a point $\neq$ x. If x is not a limit point it is called an \textbf{isolated 	  point}.
		\item A point x is an \textbf{interior point} of E if there $\exists$ r such that $N_r(x)\le E$.
		\item E is \textbf{open} if every point of E is an interior point.
		\item E is \textbf{closed} if every point of E belongs in E.
		\item E is \textbf{dense} in X if every point of X is a limit point of E, or a point of E, or both. (e.g. rationals with in real numbers)
		\item E is \textbf{bounded} if for some $r>0$, and $x \in X$, $E \subseteq N_r(x)$.
	\end{enumerate}

\end{definition}

\begin{theorem}
	Every neighborhood is an open set.
\end{theorem}

\begin{theorem}
	If X is limit point of E, then every neighborhood of X contains infinitely many points of E. 
\end{theorem}

\begin{example}
	X = $\mathbb{R}$, then (a,b) is open, [a,b] is close, (a,b] and [a,b) are neither open nor closed.
\end{example}

\begin{example}
	$X = \mathbb{R}^2$ (see sketch in notes.)
\end{example}

\begin{theorem}
	Suppose $Y \subset X$ (a metric space) and take $E \subseteq Y$, then E is open relative to Y if and ony if $E=Y\cap G$ for some open set G of X.
\end{theorem}

\begin{theorem}
	E is open if and only if its compliment is closed.
\end{theorem}

\begin{corollary}
	\begin{enumerate}[label = \alph*)]
		\item Both X and $\emptyset$ are closed.
		\item The union of finite numbers of closed sets is closed.
		\item Arbitrary intersections of closed sets is closed.
	\end{enumerate}
\end{corollary}

\begin{theorem}
	For any metric space X, we have

	\begin{enumerate}[label = \alph*)]
		\item X and $\emptyset$ are open.
		\item The intersection of a finite nuber of open sets is open. (Note: must be finite. $E_n = (-\frac{1}{n}, \frac{1}{n})$, then $\bigcap\limits^\infty E_n = \{0\}$)
		\item The union of every collection of open sets is open. 
	\end{enumerate}
\end{theorem}
% subsection metric_spaces (end)

\subsection{Compact Sets} % (fold)

\begin{definition}
	A Subset K of a metric space X is \textbf{compact} if every open  cover of k contains a finite subcover.
	This for all collections $G_\alpha, \alpha \in A$ of open sets such that $\bigcup\limits_A G_\alpha > K$ there exists a finite collection $G_{\alpha_i}, i = 1,2,\dots,$ such that $k < \cup G_\alpha$
\end{definition}

\begin{example}
	\begin{enumerate}[label = \alph*)]	
		\item X = $\mathbb{R}$, E = (0,1)\\
		Let $G_\alpha = (\frac{1}{\alpha},1), \alpha = 1, 2, \dots$\\
		Clearly , $\bigcup\limits^\infty_{\alpha = 1} G_\alpha \subset (0,1)$, but also,\\
		$K \not\subset \bigcup\limits^\infty_{\alpha = 1} G_\alpha$.
		\item $X = \mathbb{R}, E=[0,\infty)$, let $G_\alpha(-1, \alpha), \alpha \ge 1$. Then $E \subset \bigcup\limits^\infty G_\alpha$, but $E \not\subset \bigcup\limits^\infty_{\alpha = 1}, \forall n$.
	\end{enumerate}	
\end{example}

\begin{theorem}
	Suppose $k < Y < X$, (X is a metric space). Then k is a compact space with respect to Y if and only if k is a compact space of X.
\end{theorem}

\begin{proof}
	"\underline{$\Leftarrow$}" Suppose k is compact relative to X and let $V_\alpha, \alpha \in A$ be open sets relative to Y, such that $k \subset \bigcup\limits_{\alpha \in A}$. By Theorem 1.2.12 (13 in notes), $V_\alpha = Y\cap G_\alpha$, some $G_\alpha$ open relative to X. (Note: $k \subset \cup G_\alpha$.) Thus, there exists a finite subcover, $k \subset \bigcup\limits^n_{i=1} G_{\alpha_i}$. But then, 
		\begin{align*}
			k \subset Y \cap (\bigcup\limits^n_{i=1} G_{\alpha_i}) &= \bigcup\limits^n_{i=1}(Y \cap G_{\alpha_i})\\
			&= \bigcup\limits^n_{i=1} V_{\alpha_i}
		\end{align*}
	"\underline{$\Rightarrow$}" Suppose k is compact relative to Y, and let $G_\alpha, \alpha \in A$ be open relative to X, so $k \subset \bigcup\limits_{\alpha \in A}$. But then,
		\begin{align*}
		 	k \subset Y \cap (\bigcup\limits^n_{i=1} G_{\alpha_i}) &= \bigcup\limits^n_{i=1}(Y \cap G_{\alpha})\\
			&= \bigcup\limits^n_{i=1} V_{\alpha}, V_\alpha \text{ open with respect to Y.}\\
			\text{Thus, } k \subset \bigcup\limits^n_{i=1} V_{\alpha_i}) &= Y \cap (\bigcup\limits^n_{i=1} G_{\alpha})\\
		 \end{align*}
		 So, $k \subset \bigcup\limits^n_{i=1} G_{\alpha}$ 
\end{proof}

\begin{theorem}
	If k is a compact subset of a metric space, X, then k is closed and bounded.
\end{theorem}

\begin{proof}
	We'll show k is closed by showing $k^c$ is open. Let $p \in k^c$. For each $q \in k$ we will consider $N_{r_q}(q)$ where $r_q = \frac{1}{2} d(p,q)$. Since k is cmpact, there exists $(q_1, q_2, \dots, q_n) \in k$ such that $k \subset \bigcup\limits^n_{i=1} N_{r_{q_i}}(q_i)$. Let $G = \bigcap\limits^n_{i=1} N_{r_{q_i}}(p)$. (Note: $(\bigcup\limits^n N_{r_{q_i}}(q_i))\cap G = \emptyset$.)
\end{proof}

\begin{theorem}
	Closed (with repsect to X) subsets of compact sets are compact.
\end{theorem}

\begin{proof}
	Let $F \subseteq k \subseteq X$, where X is a metric space, k is compact, and F is closed with respect to X. Let $G_\alpha, \alpha \in A$, be open such that $F \subset \bigcup\limits_{\alpha \in A} G_\alpha$ (F is "covered" by $\cup G_\alpha$). F close implies $F^c$ is open. Then the collection $\{F^c, G_\alpha\}$ covers k. Let $k \subset F^c \cup G_\alpha$ which implies $F \subset \cup G_\alpha$. 
\end{proof}

\begin{theorem}
	If E is an infinite subset of a compact set k, then E has a limit point in k.
\end{theorem}

\begin{example}
	Let X be the space of rational numbers, with $d(p,d) = |p-d|$. Show that $E=\{p\in X; 2<p^2<3\}$ is closed, bounded, but not compact.
\end{example}
% subsubsection compact sets(end)
%------------------------------------------------

% \section{Citation}\index{Citation}

% This statement requires citation \cite{book_key}; this one is more specific \cite[122]{article_key}.

% %------------------------------------------------

% \section{Lists}\index{Lists}

% Lists are useful to present information in a concise and/or ordered way\footnote{Footnote example...}.

% \subsection{Numbered List}\index{Lists!Numbered List}

% \begin{enumerate}
% \item The first item
% \item The second item
% \item The third item
% \end{enumerate}

% \subsection{Bullet Points}\index{Lists!Bullet Points}

% \begin{itemize}
% \item The first item
% \item The second item
% \item The third item
% \end{itemize}

% \subsection{Descriptions and Definitions}\index{Lists!Descriptions and Definitions}

% \begin{description}
% \item[Name] Description
% \item[Word] Definition
% \item[Comment] Elaboration
% \end{description}

% %----------------------------------------------------------------------------------------
% %	CHAPTER 2
% %----------------------------------------------------------------------------------------

% \chapter{In-text Elements}

% \section{Theorems}\index{Theorems}

% This is an example of theorems.

% \subsection{Several equations}\index{Theorems!Several Equations}
% This is a theorem consisting of several equations.

% \begin{theorem}[Name of the theorem]
% In $E=\mathbb{R}^n$ all norms are equivalent. It has the properties:
% \begin{align}
% & \big| ||\mathbf{x}|| - ||\mathbf{y}|| \big|\leq || \mathbf{x}- \mathbf{y}||\\
% &  ||\sum_{i=1}^n\mathbf{x}_i||\leq \sum_{i=1}^n||\mathbf{x}_i||\quad\text{where $n$ is a finite integer}
% \end{align}
% \end{theorem}

% \subsection{Single Line}\index{Theorems!Single Line}
% This is a theorem consisting of just one line.

% \begin{theorem}
% A set $\mathcal{D}(G)$ in dense in $L^2(G)$, $|\cdot|_0$. 
% \end{theorem}

% %------------------------------------------------

% \section{Definitions}\index{Definitions}

% This is an example of a definition. A definition could be mathematical or it could define a concept.

% \begin{definition}[Definition name]
% Given a vector space $E$, a norm on $E$ is an application, denoted $||\cdot||$, $E$ in $\mathbb{R}^+=[0,+\infty[$ such that:
% \begin{align}
% & ||\mathbf{x}||=0\ \Rightarrow\ \mathbf{x}=\mathbf{0}\\
% & ||\lambda \mathbf{x}||=|\lambda|\cdot ||\mathbf{x}||\\
% & ||\mathbf{x}+\mathbf{y}||\leq ||\mathbf{x}||+||\mathbf{y}||
% \end{align}
% \end{definition}

% %------------------------------------------------

% \section{Notations}\index{Notations}

% \begin{notation}
% Given an open subset $G$ of $\mathbb{R}^n$, the set of functions $\varphi$ are:
% \begin{enumerate}
% \item Bounded support $G$;
% \item Infinitely differentiable;
% \end{enumerate}
% a vector space is denoted by $\mathcal{D}(G)$. 
% \end{notation}

% %------------------------------------------------

% \section{Remarks}\index{Remarks}

% This is an example of a remark.

% \begin{remark}
% The concepts presented here are now in conventional employment in mathematics. Vector spaces are taken over the field $\mathbb{K}=\mathbb{R}$, however, established properties are easily extended to $\mathbb{K}=\mathbb{C}$.
% \end{remark}

% %------------------------------------------------

% \section{Corollaries}\index{Corollaries}

% This is an example of a corollary.

% \begin{corollary}[Corollary name]
% The concepts presented here are now in conventional employment in mathematics. Vector spaces are taken over the field $\mathbb{K}=\mathbb{R}$, however, established properties are easily extended to $\mathbb{K}=\mathbb{C}$.
% \end{corollary}

% %------------------------------------------------

% \section{Propositions}\index{Propositions}

% This is an example of propositions.

% \subsection{Several equations}\index{Propositions!Several Equations}

% \begin{proposition}[Proposition name]
% It has the properties:
% \begin{align}
% & \big| ||\mathbf{x}|| - ||\mathbf{y}|| \big|\leq || \mathbf{x}- \mathbf{y}||\\
% &  ||\sum_{i=1}^n\mathbf{x}_i||\leq \sum_{i=1}^n||\mathbf{x}_i||\quad\text{where $n$ is a finite integer}
% \end{align}
% \end{proposition}

% \subsection{Single Line}\index{Propositions!Single Line}

% \begin{proposition} 
% Let $f,g\in L^2(G)$; if $\forall \varphi\in\mathcal{D}(G)$, $(f,\varphi)_0=(g,\varphi)_0$ then $f = g$. 
% \end{proposition}

% %------------------------------------------------

% \section{Examples}\index{Examples}

% This is an example of examples.

% \subsection{Equation and Text}\index{Examples!Equation and Text}

% \begin{example}
% Let $G=\{x\in\mathbb{R}^2:|x|<3\}$ and denoted by: $x^0=(1,1)$; consider the function:
% \begin{equation}
% f(x)=\left\{\begin{aligned} & \mathrm{e}^{|x|} & & \text{si $|x-x^0|\leq 1/2$}\\
% & 0 & & \text{si $|x-x^0|> 1/2$}\end{aligned}\right.
% \end{equation}
% The function $f$ has bounded support, we can take $A=\{x\in\mathbb{R}^2:|x-x^0|\leq 1/2+\epsilon\}$ for all $\epsilon\in\intoo{0}{5/2-\sqrt{2}}$.
% \end{example}

% \subsection{Paragraph of Text}\index{Examples!Paragraph of Text}

% \begin{example}[Example name]
% \lipsum[2]
% \end{example}

% %------------------------------------------------

% \section{Exercises}\index{Exercises}

% This is an example of an exercise.

% \begin{exercise}
% This is a good place to ask a question to test learning progress or further cement ideas into students' minds.
% \end{exercise}

% %------------------------------------------------

% \section{Problems}\index{Problems}

% \begin{problem}
% What is the average airspeed velocity of an unladen swallow?
% \end{problem}

% %------------------------------------------------

% \section{Vocabulary}\index{Vocabulary}

% Define a word to improve a students' vocabulary.

% \begin{vocabulary}[Word]
% Definition of word.
% \end{vocabulary}

% %----------------------------------------------------------------------------------------
% %	PART
% %----------------------------------------------------------------------------------------

% \part{Part Two}

% %----------------------------------------------------------------------------------------
% %	CHAPTER 3
% %----------------------------------------------------------------------------------------

% \chapterimage{chapter_head_1.pdf} % Chapter heading image

% \chapter{Presenting Information}

% \section{Table}\index{Table}

% \begin{table}[h]
% \centering
% \begin{tabular}{l l l}
% \toprule
% \textbf{Treatments} & \textbf{Response 1} & \textbf{Response 2}\\
% \midrule
% Treatment 1 & 0.0003262 & 0.562 \\
% Treatment 2 & 0.0015681 & 0.910 \\
% Treatment 3 & 0.0009271 & 0.296 \\
% \bottomrule
% \end{tabular}
% \caption{Table caption}
% \end{table}

% %------------------------------------------------

% \section{Figure}\index{Figure}

% \begin{figure}[h]
% \centering\includegraphics[scale=0.5]{placeholder}
% \caption{Figure caption}
% \end{figure}

%----------------------------------------------------------------------------------------
%	BIBLIOGRAPHY
%----------------------------------------------------------------------------------------

\chapter*{Bibliography}
\addcontentsline{toc}{chapter}{\textcolor{ocre}{Bibliography}}
\section*{Books}
\addcontentsline{toc}{section}{Books}
\printbibliography[heading=bibempty,type=book]
\section*{Articles}
\addcontentsline{toc}{section}{Articles}
\printbibliography[heading=bibempty,type=article]

%----------------------------------------------------------------------------------------
%	INDEX
%----------------------------------------------------------------------------------------

\cleardoublepage
\phantomsection
\setlength{\columnsep}{0.75cm}
\addcontentsline{toc}{chapter}{\textcolor{ocre}{Index}}
\printindex

%----------------------------------------------------------------------------------------

\end{document}